\documentclass{article}
\usepackage[utf8]{inputenc}
\usepackage[english]{babel}
\usepackage{soul}

\usepackage[usenames, dvipsnames]{color}

\newcommand{\sqdeg}{deg$^2$ }
\newcommand{\omb}{\ensuremath{\Omega_b h^2}}
\newcommand{\omc}{\ensuremath{\Omega_c h^2}}
\newcommand{\clpp}{\ensuremath{C_{L}^{\phi\phi}}}
\newcommand{\cpmf}{\ensuremath{C_{\ell}^{\rm PMF}}}

\newcommand{\cpmftens}{\ensuremath{C_{\ell}^{\rm PMF,\,tens}}}
\newcommand{\cpmfvec}{\ensuremath{C_{\ell}^{\rm PMF,\,vec}}}
\newcommand{\apmf}{\ensuremath{A_{\rm PMF}}}
\newcommand{\bpmf}{\ensuremath{B_{\rm 1\,Mpc}}}
\newcommand{\alens}{\ensuremath{A_{\rm lens}}}
\newcommand{\lcdm}{\ensuremath{\Lambda}CDM}
\newcommand{\nrun}{\ensuremath{n_{\rm run}}}
\newcommand{\neff}{\ensuremath{N_{\rm eff}}}
\newcommand{\ho}{H\ensuremath{_0}}
\newcommand{\mnu}{\ensuremath{\sum m_\nu}}
\newcommand{\ukarcmin}{\ensuremath{\mu}{\rm K-arcmin}}
\newcommand{\lknee}{\ensuremath{\ell_{\rm knee}}}
\newcommand{\fermilat}{\textit{Fermi}-LAT}

\newcommand{\be}{\begin{equation}}
\newcommand{\ee}{\end{equation}}
\newcommand{\planck}{{\sl Planck}}
\newcommand{\wmap}{{\sl WMAP}}
\newcommand{\bicepkeck}{BICEP2/Keck Array}
\newcommand{\sptnew}{SPT-3G}
\newcommand{\pb}{\textsc{Polarbear}}
\newcommand{\simons}{Simons Array}
\newcommand{\sptpol}{SPTpol}
\newcommand{\advactpol}{Adv.~ACTpol}

\newcommand{\tbd}[1]{\textcolor{Red}{{\bf TBD}: #1}}
\newcommand{\gab}[1]{\textcolor{Orchid}{[{\bf GS}: #1]}}
\newcommand{\changed}[1]{\textcolor{Red}{#1}}
\newcommand{\removed}[1]{\textcolor{Red}{}}

\begin{document}

Dear Editor,
\\

We'd like to thank the reviewer for their time, carefully reading our paper and providing their valuable feedback. Below we address their feedback and list the changes we've made to the paper in response.
\\

\textcolor{blue}{The reviewer's feedback is shown in blue.}

Our response to the feedback will be included in black. 

\textcolor{PineGreen}{The changed sections in our paper will then be included below in green \textbf{with additions presented in boldface} and removed sections presented with a strike through them.}
\\

\textcolor{blue}{The end of the 2nd paragraph says that this work constrains the possibility of PMFs "or parity-violating physics". As far as I can tell, the analysis uses only parity-even spectra, such as TT, TE, EE and BB. The authors should either clarify what they mean or remove the claim.}

In addition to producing a non-zero EB and TB power spectrum, PMFs and other parity-violating physics are also known to convert EE power into BB power. By looking for excess power in the BB power spectrum we can find the strength of PMFs or equate the effects of parity-violating physics to an equivalent PMF field strength.

\textcolor{PineGreen}{CMB B-mode measurements can also be used to constrain more exotic models, such as the possible existence of cosmic birefringence (\citep{carroll98,lue99}) or primordial magnetic fields (PMFs) (\citep{kosowsky96, seshadri01}).  
Both effects lead to the rotation of E-modes into B-modes \textbf{which leads to an increase in power in the BB spectrum.}
 By equating the magnitude of the resulting B-modes, parity-violating processes can be translated into an equivalent PMF strength.  }
\\

\textcolor{blue}{The work assumes a fixed PMF template based on a model with a fixed value of the PMF generation time: $a_{\nu}/a_{PMF}$. This parameter is entirely arbitrary, and directly determines the amplitude of the tensor mode, which is responsible for the tight bound the authors get from BICEP2/Keck. I think it is misleading to present bounds based on this assumption as bounds on PMF. The authors should, at the very least, add a discussion of how the bound changes if one allows the PMF generation time to be a free parameter that is marginalized over. In such a case, the bound will be dominated by the vector mode contribution, with SPTPol likely dominating the constraints.}

As per the advice of the reviewer we have removed any mention of the CMBACT code.

\textcolor{PineGreen}{
\st{
We have compared the calculated templates to the outputs of the earlier CMBACT-derived code.
We find excellent agreement  for the vector modes as would be hoped, however a roughly factor of two difference on the tensor mode power. 
The origin of the difference is unclear.
We have chosen to use the more recent (Zucca16) code instead of the CMBACT-derived code, largely due to our desire to extend the calculation out to $\ell=10000$ for the Fisher matrix forecasts.}
\textbf{We use the code from zucca16 which allows us to extend the calculation out to $\ell=10000$ for the Fisher matrix forecasts.}}
\\

\textcolor{PineGreen}{
We find a 95\% CL upper limit of $\bpmf < \upperBplk$\,nG for \planck{} alone,
this is somewhat better than the limit found by \citet{planck15-19} of $\bpmf < 2.0$\,nG with $n_B=-2.9$ held fixed as we have done.}
\textcolor{PineGreen}{
\st{The improvement is due to a difference in the calculated tensor PMF power (for the same parameters) between the CMBACT-derived and \citet{zucca16}}}
\textcolor{PineGreen}{
\textbf{The improvement comes from our choice of value for the PMF generation time versus the decision in \citet{planck15-19} to marginalise over PMF generation time altogether.}}
\textcolor{PineGreen}{
\st{Using the CMBACT-derived code that planck15-19 used, we find a consistent upper limit of $\bpmf < 1.9$\,nG.
Returning to} \textbf{Using} the \citet{zucca16} code and adding the ground-based polarization measurements, we find the upper limit is  reduced only slightly to $\bpmf < \upperBall$\,nG.}
\\

\textcolor{blue}{The authors report a factor of 2 difference in the normalization of the tensor mode between the public code provided by Zucca et al and a "CMBACT-derived code", saying that "the origin of the difference is unclear". The public version of CMBACT is a code for CMB anisotropies from cosmic strings and does not concern PMFs. Any code that the authors might have used would be a private modification. Either the authors should clarify what code they are referring to, or avoid mentioning it.}

\textcolor{pinegreen}{}
\\

\textcolor{blue}{In the results section, the authors claim that "The improvement [compared to Planck] is due to a difference in the calculated tensor PMF power (for the same
parameters) between the CMBACT-derived and Zucca et al. (2016) codes". The also claim that Planck used "the CMBACT-derived" code. This is a false claim. Firstly, see the point 3 above. Secondly, as far as I know, Planck used a private code written primarily by Daniela Paoletti.}

\textcolor{blue}{Instead, the "improvement" comes from the fact that the analysis by Planck collaboration (as well as Zucca et al) marginalized over the PMF generation time, which negated the contribution of the PMF tensor modes to the constraints.}

\textcolor{blue}{In summary, I think the paper contains some claims that need revisiting. I suggest repeating the analysis with the PMF generation time set free and report the bounds in that case. Only then they can be seriously compared to previous results in the literature.}



\\
\textcolor{red}{Changes by section - to dos}
\\


\changed{2.2 old}

\be
C_\ell = C_\ell^{\rm{CAMB}} + \apmf \cpmf
\ee

The calculation of the PMF template, \cpmf, is described in \S\ref{sec:template}.

For the real data, unless noted, we assume the six-parameter, spatially-flat \lcdm{} model with a single massive neutrino of 60\,meV and a seventh parameter \apmf{} describing the power due to PMFs. 

\changed{2.2 new}

\be
C_\ell = C_\ell^{\rm{CAMB}} + \apmf \left[\cpmfvec + \left(\frac{\beta}{20}\right)^{1.9}  \cpmftens \right]
\ee
The calculation of the PMF templates, \cpmfvec{} and \cpmfvec{}, and the motivation for the $\beta$ scaling are described in \S\ref{sec:template}. 
The parameter $\beta = {\rm ln}(a_{\nu}/a_{\rm{PMF}})$ relates the timing of neutrino decoupling and the generation of the PMF, where $a_x$ represents the scale factor at the respective events.

For the real data, unless noted, we assume the six-parameter, spatially-flat \lcdm{} model with a single massive neutrino of 60\,meV and \changed{2 parameters, \apmf{} and $\beta$, }describing the power due to PMFs.

There are two points to note about the PMF priors. 
First, this is a flat prior on the observed PMF power, $\apmf$, not the rms magnetic field strength, \bpmf, that has often been used in the literature. 
In the current era of upper limits, this prior choice has some impact on the resulting PMF limits. 
Second, the exact range of the uniform prior on $\beta$ matters. 
Here we follow \citet{planck15-19} and \citet{zucca16} and use $\beta \in [11.513, 41.447]$.\footnote{This corresponds to $log_{10} (a_{\nu}/a_{\rm{PMF}}) \in[4,17]$.}
However, the constraints, especially those from \planck{} alone, weaken substantially if the lower bound on $\beta$ is lowered further. 
If we follow \citet{polarbear15} who used $\beta \in [0,39]$, the upper limits are relaxed by a factor of roughly eight for \planck-only and by a factor of two for the combined dataset. 

\end{document}


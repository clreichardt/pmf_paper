\documentclass{article}
\usepackage[utf8]{inputenc}
\usepackage[english]{babel}
\usepackage{soul}

\usepackage[usenames, dvipsnames]{color}

\usepackage[table,usenames,dvipsnames]{xcolor}
%\usepackage{amsmath}
%\usepackage{subfigure}
\usepackage[backref,breaklinks,colorlinks,citecolor=blue]{hyperref}
\usepackage{natbib}
%\usepackage{natbib}
\bibliographystyle{fapj}
%\usepackage{graphicx}
%\usepackage{multirow}
\usepackage{soul}
\setlength{\parindent}{0pt}
\newcommand{\sqdeg}{deg$^2$ }
\newcommand{\omb}{\ensuremath{\Omega_b h^2}}
\newcommand{\omc}{\ensuremath{\Omega_c h^2}}
\newcommand{\clpp}{\ensuremath{C_{L}^{\phi\phi}}}
\newcommand{\cpmf}{\ensuremath{C_{\ell}^{\rm PMF}}}

\newcommand{\cpmftens}{\ensuremath{C_{\ell}^{\rm PMF,\,tens}}}
\newcommand{\cpmfvec}{\ensuremath{C_{\ell}^{\rm PMF,\,vec}}}
\newcommand{\apmf}{\ensuremath{A_{\rm PMF}}}
\newcommand{\bpmf}{\ensuremath{B_{\rm 1\,Mpc}}}
\newcommand{\alens}{\ensuremath{A_{\rm lens}}}
\newcommand{\lcdm}{\ensuremath{\Lambda}CDM}
\newcommand{\nrun}{\ensuremath{n_{\rm run}}}
\newcommand{\neff}{\ensuremath{N_{\rm eff}}}
\newcommand{\ho}{H\ensuremath{_0}}
\newcommand{\mnu}{\ensuremath{\sum m_\nu}}
\newcommand{\ukarcmin}{\ensuremath{\mu}{\rm K-arcmin}}
\newcommand{\lknee}{\ensuremath{\ell_{\rm knee}}}
\newcommand{\fermilat}{\textit{Fermi}-LAT}

\newcommand{\be}{\begin{equation}}
\newcommand{\ee}{\end{equation}}
\newcommand{\planck}{{\sl Planck}}
\newcommand{\wmap}{{\sl WMAP}}
\newcommand{\bicepkeck}{BICEP2/Keck Array}
\newcommand{\sptnew}{SPT-3G}
\newcommand{\pb}{\textsc{Polarbear}}
\newcommand{\simons}{Simons Array}
\newcommand{\sptpol}{SPTpol}
\newcommand{\advactpol}{Adv.~ACTpol}

\newcommand{\tbd}[1]{\textcolor{Red}{{\bf TBD}: #1}}
\newcommand{\gab}[1]{\textcolor{Orchid}{[{\bf GS}: #1]}}
\newcommand{\changed}[1]{\textcolor{Red}{#1}}
\newcommand{\removed}[1]{\st{#1}}
\newcommand{\added}[1]{\textbf{#1}}
\newcommand{\reviewer}[1]{\textcolor{Blue}{#1}}
\newcommand{\diff}[1]{\textcolor{PineGreen}{#1}}
\newcommand{\hide}[1]{\iffalse{#1}}

%file with numbers from chains or fisher matrices
%pulled out for ease of filling in


% numbers from 2017/04/02 chains with beta free within planck range
\newcommand{\upperAall}{ 0.36}
\newcommand{\upperAplk}{ 0.76}
\newcommand{\upperAplkalens}{ 0.64}
\newcommand{\upperAallalens}{ 0.30}
\newcommand{\upperAsptpol}{ 0.45}
\newcommand{\upperAbicep}{ 0.44}

\newcommand{\upperAallalensr}{ 0.25}
\newcommand{\medAlensall}{ 1.123}
\newcommand{\sigmaAlensall}{ 0.064}
\newcommand{\medAlensRall}{ 1.127}
\newcommand{\sigmaAlensRall}{ 0.063}
\newcommand{\upperRall}{ 0.07}

%using imp sampling
%\newcommand{\upperBplk}{ 1.7}
%\newcommand{\upperBall}{ 1.6}
%using dedicated chains:
\newcommand{\upperBplk}{ 1.8}
\newcommand{\upperBall}{ 0.91}


%numbers from 2017/04/02 chains with beta free within PB range
%\newcommand{\upperAall}{ 0.66}
%\newcommand{\upperAplk}{ 5.76}
%\newcommand{\upperAplkalens}{ 3.68}
%\newcommand{\upperAallalens}{ 0.51}
%\newcommand{\upperAsptpol}{ 0.73}
%\newcommand{\upperAbicep}{ 6.13}
%\newcommand{\upperBplk}{ 3.37}
%\newcommand{\upperBall}{ 1.91}
%\newcommand{\upperAallalensr}{ 0.55}
%\newcommand{\medAlensall}{ 1.121}
%\newcommand{\sigmaAlensall}{ 0.064}
%\newcommand{\medAlensRall}{ 1.120}
%\newcommand{\sigmaAlensRall}{ 0.063}
%\newcommand{\upperRall}{ 0.07}


% numbers from chains:
%updated 30 Jan 2017
%using chains from chains_pmf/chains_20170103/
%comments are below with tmp chain numbers. this print out didn't print comments:
%\newcommand{\upperAall}{ 0.22}
%\newcommand{\upperAplk}{ 0.39}
%\newcommand{\upperAplkalens}{ 0.39}
%\newcommand{\upperAallalens}{ 0.20}
%\newcommand{\upperAsptpol}{ 0.35}
%\newcommand{\upperAbicep}{ 0.22}
%\newcommand{\upperBplk}{ 1.7}
%\newcommand{\upperBall}{ 1.6}
%\newcommand{\upperAallalensr}{ 0.17}
%\newcommand{\medAlensall}{ 1.126}
%\newcommand{\sigmaAlensall}{ 0.063}
%\newcommand{\medAlensRall}{ 1.126}
%\newcommand{\sigmaAlensRall}{ 0.062}
%\newcommand{\upperRall}{ 0.07}

%\newcommand{\upperAall}{0.22} %95% LCDM + all exp.
%\newcommand{\upperAplk}{0.40} %95% LCDM + planck
%\newcommand{\upperAplkalens}{0.38}%LCDM alens+ planck
%\newcommand{\upperAallalens}{0.20}%LCDM alens + all exp.
%\newcommand{\upperAsptpol}{0.34}%LCDM  planck+sptpol
%\newcommand{\upperAbicep}{0.21}%LCDM  planck+bicep2
%\newcommand{\upperBplk}{0.} %95% B1mpc; LCDM planck
%\newcommand{\upperBall}{0.} %95% B1mpc; LCDM planck+all
%\newcommand{\upperAallalensr}{0.17}

%Alens numbers
%\newcommand{\medAlensall}{1.122}%LCDM+ALens; with all
%\newcommand{\sigmaAlensall}{0.065}%LCDM+ALens; with all
%\newcommand{\medAlensRall}{1.126}%LCDM+R+ALens; with all
%\newcommand{\sigmaAlensRall}{0.064}%LCDM+R+ALens; with all
%r upper limits
%\newcommand{\upperRall}{0.07} %LCDM+R+ALens; with all



%numbers from fisher:
%these include beta too
\newcommand{\fisherAplk}{0.17} %LCDM w Planck

\newcommand{\fisherAlooseNbOneplk}{0.38} %LCDM11, nb=-1 w Planck
\newcommand{\fisherAlooseNbTwoplk}{\ensuremath{3.0\times10^{-6}}} %LCDM11, nb=2 w Planck
\newcommand{\fisherAlooseNbOneSThree}{\ensuremath{2.3\times10^{-2}}} %LCDM11, nb=-1 w Planck+S3
\newcommand{\fisherAlooseNbTwoSThree}{\ensuremath{1.5\times10^{-7}}} %LCDM11, nb=2 w Planck+S3

\newcommand{\fisherAlooseSFour}{\ensuremath{6.3\times10^{-3}}} %LCDM11, nb=2 w Planck+S$
\newcommand{\fisherAlooseNbOneSFour}{\ensuremath{7.4\times10^{-3}}} %LCDM11, nb=-1 w Planck+S4
\newcommand{\fisherAlooseNbTwoSFour}{\ensuremath{5.0\times10^{-8}}} %LCDM11, nb=2 w Planck+S4


\begin{document}

Dear Editor,\\

We thank the reviewer for their  careful read of our paper, and their comments and suggestions. Below we address their feedback and list the changes we've made to the paper in response.\\


The reviewer's comments are shown in \reviewer{blue,} our response is in black, and the revised text in the paper is in \diff{green}. \\

\section{Major comments}


\subsection{$\beta$ should be marginalized over}
\begin{itemize}
\item \reviewer{The work assumes a fixed PMF template based on a model with a fixed value of the PMF generation time: $a_{\nu}/a_{PMF}$. This parameter is entirely arbitrary, and directly determines the amplitude of the tensor mode, which is responsible for the tight bound the authors get from BICEP2/Keck. I think it is misleading to present bounds based on this assumption as bounds on PMF. The authors should, at the very least, add a discussion of how the bound changes if one allows the PMF generation time to be a free parameter that is marginalized over. In such a case, the bound will be dominated by the vector mode contribution, with SPTPol likely dominating the constraints.}

\item \reviewer{In summary, I think the paper contains some claims that need revisiting. I suggest repeating the analysis with the PMF generation time set free and report the bounds in that case. Only then they can be seriously compared to previous results in the literature.}
\end{itemize}

We agree with the referee that freeing the PMF generation time would strengthen the results, and have implemented this in both the fits to the real data and the Fisher matrix forecasts. All the numbers and plots have been updated with results that are marginalized over $\beta$. 
This significantly changes the results for the real data, making SPTpol more competitive with BICEP2/Keck as the referee expected. 
The impact on the Fisher forecasts however is fairly minor.\\

We have updated the text in light of this change. There are significant additions to Sec 2.2 (Methods) and 2.3 (PMF template) to describe what happens when $\beta$ is varied. The text of Sec 3 has also been updated to reflect the new upper limits (with $\beta$ marginalized over), and the lesser relative importance of BICEP2/Keck.\\

Additionally, there is a small change in the first paragraph of 4.2 to describe how the prior placed on $\beta$ in the Fisher analysis. There are also a number of places that had referred to 6 and 11 parameter models that now refer to 7 and 12 parameter models. \\\\



\subsection{CMBACT / Planck PMF code}
\reviewer{The authors report a factor of 2 difference in the normalization of the tensor mode between the public code provided by Zucca et al and a "CMBACT-derived code", saying that "the origin of the difference is unclear". The public version of CMBACT is a code for CMB anisotropies from cosmic strings and does not concern PMFs. Any code that the authors might have used would be a private modification. Either the authors should clarify what code they are referring to, or avoid mentioning it.\\
In the results section, the authors claim that "The improvement [compared to Planck] is due to a difference in the calculated tensor PMF power (for the same parameters) between the CMBACT-derived and Zucca et al. (2016) codes". They also claim that Planck used "the CMBACT-derived" code. This is a false claim. Firstly, see the point 3 above. Secondly, as far as I know, Planck used a private code written primarily by Daniela Paoletti.\\
Instead, the "improvement" comes from the fact that the analysis by Planck collaboration (as well as Zucca et al) marginalized over the PMF generation time, which negated the contribution of the PMF tensor modes to the constraints.}
\\\\

We had indeed mistakenly attributed CMBACT to Planck. 
The private CMBACT-derived code was used in the Polarbear PMF paper (Ade et al. Phys. Rev. D 92, 123509 (2015)), and did have the factor of two difference in normalization that mimicked the change in reported upper limits due to varying beta. 
We mis-identified this as the cause for the difference in the reported Planck limits.\\

We have now struck all mention of CMBACT. There is no longer a discussion of the differences in the Planck numbers since $\beta$ floats in both works.





\section{Minor comments:}
\subsection{Parity-violating physics}
\reviewer{The end of the 2nd paragraph says that this work constrains the possibility of PMFs "or parity-violating physics". As far as I can tell, the analysis uses only parity-even spectra, such as TT, TE, EE and BB. The authors should either clarify what they mean or remove the claim.}
\\\\

In addition to producing a non-zero EB and TB power spectrum, PMFs and other parity-violating physics are also known to convert EE power into BB power. By looking for excess power in the BB power spectrum we can find the strength of PMFs or equate the effects of parity-violating physics to an equivalent PMF field strength.\\\\

\diff{CMB B-mode measurements can also be used to constrain more exotic models, such as the possible existence of cosmic birefringence (Carroll 1998; Lue et al. 1999) or primordial magnetic fields (PMFs) (Kosowsky \& Loeb
1996; Seshadri \& Subramanian 2001).  
Both effects lead to the rotation of E-modes into B-modes \textbf{which leads to an increase in power in the BB spectrum.}
By equating the magnitude of the resulting B-modes, parity-violating processes can be translated into an equivalent PMF strength.}\\

\end{document}

